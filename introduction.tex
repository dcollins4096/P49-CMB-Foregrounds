\section{Introduction}
The theory of inflation is a widely accepted theory of the origins of the early
universe in which space underwent a period of extremely rapid expansion
shortly after the big bang. This theory has been extremely successful in
explaining a wide range of cosmological problems yet direct confirmation thus
far has been elusive. Primordial gravitational waves created during the
inflationary epoch \citep{Starobinskii79, Kamionkowski15} and stretched to observable scales would imprint curl-like
B-modes on the polarized CMB \citep{Guth81, Linde82} and provide just such a direct confirmation.

Direct observation of these polarized signals are complicated by the presence of
the polarized dust foreground of the ISM and many recent efforts
\citep{Kritsuk17, Bracco19} have been
devoted to better understanding the properties of this foreground in order to
achieve the sensitivities needed to measure the primordial B-mode signal
characterized by the tensor-to-scalar ratio $r$ \citep{Baumann09, Planck15}. One
of the more promising results found by Planck's 353GHz waveband survey \citep{Planck15, Planck18XI} is that
the polarized dust foreground shows several consistent properties across the
high lattitude sky. In this paper we focus on exploring the observations that the ratio of
E-mode to B-mode power $r$ in the foreground dust spectra is ~2
\citep{Caldwell16, Planck18XI} and that both E and
B power spectra have a measured slope of ~-2.5 (specifically $\alpha_{EE}=-2.42$ and
$\alpha_{BB}=-2.54$) \citep{Planck15, Planck18XI}. We also
discuss the finding by \citep{Planck18XI} that the predicted T/E and T/B power
ratio of 0 is observed to be non-zero.

A linear polarization signal can be decomposed into two rotationally invariant
quantities, E (gradient/scalar) and B (curl/tensor) modes \citep{Kamionkowski97,
Zaldarriaga97}. Since random orientated or random amplitude fixed orientation
polarization signals would appear with equal E and B powers, the Planck finding
of an EE/BB power ratio ~2 implies some non-random coherence between the two signals
that must arise from underlying physics in the dust foreground.

The slopes of the power spectra reflect the amount of power at a given size
scale in the turbulent medium as energy cascades from the injection scale to the
dissipation scale (this gap being called the inertial range) where energy is
dissipated into the surrounding system, typically in the form of heat due to
viscocity. Several relatively recent works that explore these effects in more
detail are \citep{Clark14,Ghosh17,Cho02}. The slope of such a cascade in
the inertial range is dictated by several factors such as the sonic Mach number
$M_{s}$ which is the ratio of flow velocity to the speed of sound in the gas and
the Alfven Mach number $M_{a}$ which is the ratio of the flow velocity to the
Alfven velocity. The Alfven velocity is the speed of a particular type of
oscillation of plasma ions in a magnetic field and is dictated by the magnetic
field strength and the gas density. Finding the correct combination of Mach
number and Alfven Mach number to allow for spectral slopes in E and B of
$\alpha=~-2.5$ \citep{Planck18XI} reveals important information on the properties of the polarized foreground.
