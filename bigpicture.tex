\section{Conclusions}

Our simulations provide many insights into the possible nature of the polarized
dust foreground. We learned precisely how the slopes of the EE and BB power
spectra and EE/BB power ratios behave with respect to changing sonic and Alfven
mach numbers as well as observation orientation with respect to the background
magnetic field. We find that these values become increasingly non-isotropic with
respect to line of sight as the flow becomes sub-Alfvenic. 

These observations combined with a detailed parameter space analysis suggest that the
\citep{Planck18XI} results for EE and BB slopes $\alpha_{xx} \approx -2.5$ and
E/B power ratio $r \approx 2.0$ can most easily be explained by a super-sonic,
super-Alfvenic flow. If the dust is indeed sub-Alfvenic it will require a much
more super-sonic (or much more sub-Alfvenic) flow than those probed in this
project and the exact values of $M_s$ and $M_a$ able to probe the desired
parameter space would be much more strict.

In principle this narrowing of the allowed parameter space could be a good thing
since it would provide much more exact information into the allowed properties
of the polarized dust foreground, exactly what we are hoping to find. However
our results seem to suggest for now that so long as the flow is at least
moderately super-Alfvenic and super-sonic, a wide range of values for $M_s$ and
$M_a$ satisfy the observed parameters and isotropic nature.

Ultimately since a highly super-sonic flow in nature for these dust clouds is not unlikely,
it would be a valuable endeaver to run another suite of similar simulations with
greatly super-sonic, sub-Alfvenic conditions to see if these values are even
capable of achieving the desired parameter space our results suggest that they
might. 
