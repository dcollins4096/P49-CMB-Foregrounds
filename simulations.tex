\section{Simulations and Analysis}

\subsection{MHD Equations}
These simulations are done using the open source code Enzo (\citet{Bryan14}) to
solve the Eulerian equations of ideal magnetohydrodynamics (MHD) using the
method of adaptive mesh refinement (AMR) to achieve the dynamic resolution scales
needed for this type of problem. These MHD equations (as can be found in the
Enzo documentation cited above) with no gravity are shown below in cgs
units with magnetic permeability set to 1.   

\begin{equation} \label{eq:MHD1}
    \frac{\partial \rho}{\partial t}
+ \nabla \cdot
(\rho \mathbf{v})
= 0
\end{equation}
\begin{equation} \label{eq:MHD2}
    \frac{\partial \rho \mathbf{v} }{\partial t}
+ \nabla \cdot
\left(
        \rho \mathbf{v} \mathbf{v}
        + \mathbf{I} P
        - \frac { \mathbf{B} \mathbf{B} }{ 8 \pi }
\right)
=
0
\end{equation}
\begin{equation} \label{eq:MHD3}
\frac{\partial E}{\partial t}
+ \nabla \cdot
\left[
        (E + P) \mathbf{v}
        -
        \frac
        { \mathbf{B} ( \mathbf{B} \cdot \mathbf{v} ) }
        { 4 \pi }
\right]
= 0
\end{equation}
\begin{equation} \label{eq:MHD4}
\frac {\partial \mathbf{B} }{\partial t}
- \nabla \times ( \mathbf{v} \times \mathbf{B})
=0
\end{equation}


where $\mathbf{v}\mathbf{v}$ and $\mathbf{B}\mathbf{B}$ are
the velocity and the magnetic field outer products, $\phi$ is
the density,$\Lambda$ and $\Gamma$ represent
radiative cooling and heating and $\mathbf{F}_{cond}$ is the flux due to thermal
heat conduction.
The total energy density  is
equal to the total of the kinetic,
the thermal and the magnetic energy,

\begin{equation} \label{eq:E-total}
E = e + \frac{ \rho v^2 }{ 2 } + \frac{B^2}{8\pi}
\end{equation}

and

\begin{equation} \label{eq:P-total}
P = p + \frac{B^2}{8\pi}
\end{equation}

is the total thermal plus the magnetic pressure.

The system is closed by an equation of state.

\begin{equation} \label{eq:EOS}
e = \frac{p}{\gamma - 1}
\end{equation}

where the equation of state is shown for an ideal gas with a ratio of specific
heats $\gamma$.
\subsection{Adaptive Mesh Refinement}
The difficult problem mentioned before of large dynamic spatial and temporal
scales inherent to this study is handled through the implementation of adaptive
mesh refinement (AMR). This method actively increases and decreases the 
local mesh resolution and time step based on desired local parameters such as
Jean's length, metalicity or density. This enables us to conserve computational
power on regions that are largely static and direct it toward the dynamic, dense
regions that are evolving rapidly on the galactic time scales. This method often
enables us to achieve effective resolutions several orders of magnitude larger
than would be possible using a fixed grid with a similar computational time
while losing very little accuracy in the coarse regions. 

When running our simulation we select several criteria (such as those outlined
above) that will 'flag' a region as requiring increased grid resolution or
shorter time steps in order to achieve the desired accuracy. This region is then
subdivided into several smaller regions that the equations can be solved on. If
the smaller regions no longer satisfy the 'flagged' criterion they are solved
and the entire simulation moves forward. If any zone within this new region is
still being flagged then it too is subdivided and rechecked until either an
appropriate resolution is achieved or it reaches the maximum resolution defined
manually in the code at which point it is solved as is. 

\subsection{Simulation Parameters}

These simulations use a root grid of $512^{3}$ with two levels of refinement to
yield an effective resolution of $2048^{3}$. The sonic Mach number $M_{s}$ and
Alfven Mach number $M_{a}$ are chosen here to be $$M_{s} = 0.5, 1, 2, 3$$ and
$$M_{a} = 0.5, 1, 2$$ across the suite of 12 simulations designed to probe the
polarized dust parameter space.

Here I need to list all the important parameter values such as $\rho_0$ the
ratio of solenoidal to compressible turbulence, gamma, blah blah.

The box is driven by ??purely solenoidal turbulence?? throughout the simulation and
allowed to run for 10 crossing times $t_{cross}$. The data presented in this
paper is taken from times between 5 to 10 $t_{cross}$ to ensure that a
statistically relaxed state has been reached before analysis is begun.

\subsection{Synthetic Observations}

When we speak of synthetic measurements we are referring to the ability to
use simulations (or numerical methods in general) to reproduce measurements or
data in the same way that it is collected and analyzed by physical observers.
For instance this may refer to generating 2-D projected column density maps
based on a luminosity distance relationship as opposed to looking at the 3-D
density field value at a given point. It should come as no surprise that the
majority of synthetic measurements will involve the observation of sources in
emission. There are however some cases of interest to us such as extinction
mapping in the ISM where absorption maps will also play an important role.

In this work we analyze synthetic polarization dust maps in order to gain a
better insight into the relationship between the underlying physical
information, and it presents itself to observers. This is important since
ultimately we are comparing our results with values seen on the real sky.
